%%%%%%%%%%%%%%%%%%%%%%% file typeinst.tex %%%%%%%%%%%%%%%%%%%%%%%%%
%
% This is the LaTeX source for the instructions to authors using
% the LaTeX document class 'llncs.cls' for contributions to
% the Lecture Notes in Computer Sciences series.
% http://www.springer.com/lncs       Springer Heidelberg 2006/05/04
%
% It may be used as a template for your own input - copy it
% to a new file with a new name and use it as the basis
% for your article.
%
% NB: the document class 'llncs' has its own and detailed documentation, see
% ftp://ftp.springer.de/data/pubftp/pub/tex/latex/llncs/latex2e/llncsdoc.pdf
%
%%%%%%%%%%%%%%%%%%%%%%%%%%%%%%%%%%%%%%%%%%%%%%%%%%%%%%%%%%%%%%%%%%%


\documentclass[runningheads,a4paper]{llncs}

\usepackage{graphicx}
\usepackage{amssymb}
\setcounter{tocdepth}{3}
\usepackage{amsmath}
\usepackage{amssymb}
\usepackage{color}
\usepackage{tikz}
\usepackage{pgfplots}
\usepackage{caption}
\usepackage{subcaption}
\usepackage{comment}
\usepackage{booktabs}
\usepackage{nopageno}

\usepackage[algo2e, noend, noline, linesnumbered]{algorithm2e}
%\DontPrintSemicolon

\makeatletter
\newcommand{\pushline}{\Indp}% Indent
\newcommand{\popline}{\Indm}
\makeatother
\DeclareMathOperator{\pess}{pess}
\DeclareMathOperator{\opti}{opti}
\newcommand{\argmax}{\operatornamewithlimits{argmax}}

\captionsetup{compatibility=false}



%\pgfplotsset{compat=newest}
\usetikzlibrary{arrows,shapes,petri}

\newcommand{\bE}{\mathbb{E}}
\newcommand{\cA}{\mathcal{A}}
\newcommand{\cC}{\mathcal{C}}
\newcommand{\cD}{\mathcal{D}}
\newcommand{\cI}{\mathcal{I}}
\newcommand{\cN}{\mathcal{N}}
\newcommand{\cO}{\mathcal{O}}
\newcommand{\cS}{\mathcal{S}}
\newcommand{\cT}{\mathcal{T}}
\newcommand{\cZ}{\mathcal{Z}}
\newcommand{\eg}{{\it e.g.,}~}
\newcommand{\ie}{{\it i.e.,}~}

\newcommand{\redbold}[1]{\textbf{\color{red}#1}} 
\newcommand{\markw}[1]{\textbf{\color{red} /* #1 (markw) */}} 
\newcommand{\marcl}[1]{\textbf{\color{red} /* #1 (marcl) */}} 

\usepackage{url}
\urldef{\emails}\path|{marc.lanctot,m.winands}@maastrichtuniversity.nl,lisy@agents.fel.cvut.cz|
%\urldef{\mailsb}\path|anna.kramer, leonie.kunz, christine.reiss, nicole.sator,|
%\urldef{\mailsc}\path|erika.siebert-cole, peter.strasser, lncs}@springer.com|    
\newcommand{\keywords}[1]{\par\addvspace\baselineskip
\noindent\keywordname\enspace\ignorespaces#1}

\begin{document}

\mainmatter  % start of an individual contribution

% first the title is needed
\title{Modeling Extreme Values in Monte Carlo Tree Search}

% a short form should be given in case it is too long for the running head
%\titlerunning{Monte Carlo Tree Search in Simultaneous Move Games with Applications to Goofspiel}
\titlerunning{ }

% the name(s) of the author(s) follow(s) next
%
% NB: Chinese authors should write their first names(s) in front of
% their surnames. This ensures that the names appear correctly in
% the running heads and the author index.
%
\author{ }
%
\authorrunning{ }
% (feature abused for this document to repeat the title also on left hand pages)

% the affiliations are given next; don't give your e-mail address
% unless you accept that it will be published
% Agent Technology Center, Dept. of Computer Science, FEE, Czech Technical University in Prague
% 

%\institute{$^1$Department of Knowledge Engineering, \hspace{1cm} $^2$Department of Computer Science\\
%\hspace{0.5cm}Maastricht University, Netherlands \hspace{1.1cm} Czech Technical University in Prague \\
%\emails\\
%}

%Viliam Lis{\' y}
%Agent Technology Center, Dept. of Computer Science and Engineering, FEE, Czech Technical University in Prague

%
% NB: a more complex sample for affiliations and the mapping to the
% corresponding authors can be found in the file "llncs.dem"
% (search for the string "\mainmatter" where a contribution starts).
% "llncs.dem" accompanies the document class "llncs.cls".
%

%\toctitle{}
%\tocauthor{Authors' Instructions}

\maketitle


\begin{abstract}
Abstract
\end{abstract}

\section{Introduction}

\section{Background}

\subsection{Extreme Value Theory}

\subsection{Monte Carlo Tree Search}

\section{MCTS Based on Extreme Value Theory}

\section{Experiments}

\section{Conclusion}




\bibliographystyle{splncs03}
\bibliography{evt-mcts}

\end{document}
